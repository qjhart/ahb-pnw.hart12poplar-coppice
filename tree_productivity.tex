\begin{tabularx}{\linewidth}{|c|X|c|}
  \hline
  Parameter & \centering{Source} & Value\\
  \hline
  $y$ & Assimilation use efficiency.  Used in calculation of the \ac{NPP}. & 0.47\\
  \hline  kG [\kilo\reciprocal\pascal] & Determines the response of the canopy conductance to the vapor pressure deficit. & 0.5\\
  alpha [\kilogram\per\mole] & Canopy quantum efficiency. & 0.06\\
  \acs{BLcond} [\reciprocal\meter] & Canopy boundary layer conductance. Used in the calcuation of transpiration & 0.2\\
  \hline
  \rowcolor{gray9}
  \acs{LAI} & \acl{LAI} is used in the calculations of productivity, transpiration, and rainfll interception.  Defined as function of a canopy extinction coffecient and the \ac{SLA} of the tree.&\\
  \acs{k}  & \acl{k} & 0.5\\
  \acs{SLA} [$\squaren\meter\per\kilogram$] & \acl{SLA}.  Defined as a function of the tree age.  Used in the calculation of LAI. &\\
  $f0_{\acs{SLA}}$ & \acs{SLA} at initial time & 10.8\\
  $f1_{\acs{SLA}}$ & \acs{SLA} at infinite timestep & 10.8\\
  $tm_{\acs{SLA}}$ [y] & Time in years where value is the average of $f0$ and $f1$ & 1\\
  $n$  & $n>=1$; Parameter specifing the rate of change around $tm$.  $n=1$ is approximately a linear change, as n increases, change becomes more localized around $tm$. & 2\\
  \hline
  fullCanAge [yr] & Year where tree reaches full Canopy Cover. Used along with \ac{LAI} in the productivity calculation & 1.5  \\
  \hline
  \rowcolor{gray92}
  $Cond$ [$\meter\per\second$] &  Canopy Conductance.  Along with a Physiological modifer, specifies the canopy conductance.  Used in calculation of transpiration & \\
  $mn_{Cond}$ & Minimum value, when $lai=0$ & 0.0001\\
  $mx_{Cond}$ & Maximum value & 0.02\\
  $lai_{Cond}$ [$\squaren\meter\per\squaren\meter$] & \ac{LAI} where parameter reaches a maximum value. & 3.33\\
  \hline
  \rowcolor{gray9}
  $Intcptn$ [frac] & Rainfall interception fraction.  A linear function w.r.t. \ac{LAI} & \\
  $mn_{Intcptn}$ & Minimum value, when lai=0 & 0\\
  $mx_{Intcptn}$ & Maximum value & 0.15\\
  $lai_{Intcptn}$ [$\squaren\meter\per\squaren\meter$] & \ac{LAI} where parameter reaches a maximum value. & 5\\
  \hline
\end{tabularx}

