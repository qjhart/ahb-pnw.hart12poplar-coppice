% Template for PLoS
% Version 1.0 January 2009

\documentclass[10pt]{article}

% amsmath package, useful for mathematical formulas
\usepackage{amsmath}
% amssymb package, useful for mathematical symbols
\usepackage{amssymb}

% graphicx package, useful for including eps and pdf graphics
% include graphics with the command \includegraphics
\usepackage{graphicx}

% Need to have subfigures
\usepackage{subcaption}
\pdfpagebox5

% cite package, to clean up citations in the main text. Do not remove.
\usepackage{cite}

\usepackage{color}
\usepackage[squaren]{SIunits} 
%\usepackage{units}

% Use doublespacing - comment out for single spacing
%\usepackage{setspace} 
%\doublespacing


% Text layout
\topmargin 0.0cm
\oddsidemargin 0.5cm
\evensidemargin 0.5cm
\textwidth 16cm 
\textheight 21cm

% Bold the 'Figure #' in the caption and separate it with a period
% Captions will be left justified
\usepackage[labelfont=bf,labelsep=period,justification=raggedright]{caption}

% Use the PLoS provided bibtex style
\bibliographystyle{plos2009}

% Remove brackets from numbering in List of References
\makeatletter
\renewcommand{\@biblabel}[1]{\quad#1.}
\makeatother


% Leave date blank
\date{}

\pagestyle{myheadings}
%% ** EDIT HERE **

%% ** EDIT HERE **
%% PLEASE INCLUDE ALL MACROS BELOW
\usepackage[printonlyused]{acronym}
\usepackage{multicol}

%% END MACROS SECTION

\begin{document}

% Title must be 150 characters or less
\begin{flushleft}
{\Large
\textbf{Modeling Poplar Growth as a Short Rotation Woody Crop for Biofuels}
}
% Insert Author names, affiliations and corresponding author email.
\\
Quinn Hart$^{1,\ast}$,
Olga Prelipova$^{2}$,
Justin Merz$^{1}$,
Peter Tittmann$^{3}$, 
Bryan Jenkins$^{4}$
\\
$^{\textbf{1}}$ Department of Land, Air, and Water, University of Califonia, Davis, USA
$^{\textbf{2}}$ Department of Computer Science, University of Califonia, Davis, USA
$^{\textbf{3}}$ Department of Forestry, University of Califonia, Berkeley, USA
$^{\textbf{4}}$ Energy Institute, University of Califonia, Davis, USA
\\
%\\textbf{3} Author3 Dept/Program/Center, Institution Name, City, State, Country
$\ast$ E-mail: qjhart@ucdavis.edu
\end{flushleft}

% Please keep the abstract between 250 and 300 words
\section*{Abstract}

Being able to predict the growth and yield of these crop under various
environmental conditions is an important step in the development of
energy system that can incorporate these \ac{SRWC} as feedstocks.  In
this study, the \acf{3pg} model was modified for \ac{SRWC}, with the
inclusion of a coppicing model.  The model was then tested against a
number of previous studies found in the literature.  

% Please keep the Author Summary between 150 and 200 words
% Use first person. PLoS ONE authors please skip this step. 
% Author Summary not valid for PLoS ONE submissions.   
%\section*{Author Summary}

\section*{Introduction}

The goals of this research were to develop modifications to the
\ac{3pg} model to allow for an accurate representation of growth of
\ac{SRWC} over a range of environmental conditions, and to validate
the model compared to various field trials.

We are developing parameters to use the \acf{3pg} model as a means to
estimate poplar yields for the Pacific Northwest.  The model includes
basic processes for forest growth.  Modeling the physiological growth
is advantageous because it allows variation of poplar species
parameters and management practices.  Because it is a canopy carbon
balanced model, allocations for both above and below ground biomass
can be tracked for studies like life-cycle analysis.

The original \ac{3pg} model does not include coppicing as a management
practice, which is problematic as it cannot reasonably account for
post-coppicing regrowth.  The extended model includes coppicing with a
general model that allows a monthly growth contribution from an
existing root mass.  The model specifies a relatively small
contribution of aboveground growth from the accumulated root mass
after coppicing in order to initiate the next cycle of production.

The primary modification is the inclusion of a coppicing model for the
prediction of biomass production over the course of a number of
coppiing cycles.  The coppicing model introduced is a simple extension
that models the sprouting of the coppice, contribution to growth from
the existing root system, and modifications to the allocation of
resources based on the coppicing.



\begin{multicols}{2}[\section*{Acronyms}]
\addcontentsline{toc}{section}{Acronyms}
%\renewcommand{\baselinestretch}{1.0}
{\normalsize
\raggedright
%\setlength{\columnseprule}{1pt}
\begin{acronym}
\acro{3pg}[\textsc{3PG}]{Physiological Principles in Predicting Growth}
\acro{dW}[\ensuremath{\Delta W}]{Total Monthly Growth}
\acro{GIS}[\textsc{GIS}]{Geograhical Information System}
\acro{SRWC}[\textsc{SRWC}]{Short Rotation Woody Crops}
\acro{NPP}[\ensuremath{NPP}]{Net Primary Productivity}
\acro{LAI}[\ensuremath{LAI}]{Leaf Area Index}
\acro{RP}[\ensuremath{RP}]{Root Productivity}
\acro{NPPres}[\ensuremath{NPP_{res}}]{Residual desired $NPP$}
\acro{NPPt}[\ensuremath{NPP_{T}}]{$NPP$ if $LAI = LAI_{T}$}
\acro{dRdef}[\ensuremath{\Delta R_{def}}]{Residual root}
\acro{pRx}[\ensuremath{p_{R\%x}}]{Maximum root \%}
\acro{Rdp}[\ensuremath{R_{\Delta\%}}]{Root contribution}
\acro{W}[\ensuremath{W}]{Total plant mass}
\acro{WR}[\ensuremath{W_R}]{Total root mass}
\acro{fR}[\ensuremath{f_R}]{Root conversion efficiency}
\acro{fi}[\ensuremath{f_i}]{Generic growth limiters}
\end{acronym}
}
\end{multicols}

% You may title this section "Methods" or "Models". 
% "Models" is not a valid title for PLoS ONE authors. However, PLoS ONE
% authors may use "Analysis" 
\section*{Models}

\subsection*{Potential Growth Model}

Figure~\ref{fig:growth-model} shows the modeling path used in the
development of the poplar growth model. The model developed shows the
total potential of the grieed

\begin{figure}[!ht]

\caption{ \textbf{\ac{3pg} Overview.} This graph shows
  the processing steps used to develop the potential growth model }
\label{fig:grow}
\end{figure}

The \acf{3pg} model~\cite{landsberg2010physiological} takes as inputs
weather data, site factors, initial conditions, management practices,
and species information.

The model runs at a monthly timestep. At each step the physiological
parameters are calculated and carried forward to the next month. Input
weather parameters are specified at the same monthly timestep, and
additional management parameters, such as when the poplar is coppiced
are also be included.

The physiological components of the \ac{3pg} model comprises of
submodules for estimating growth modifiers and \ac{NPP}, biomass
allocation, and soil water balance.  We have added an additional
module for regrowth after
coppicing~(Section~\ref{sec:coppicing-model}).

% The first step in the modeling of the poplar growth was to develop an
% estimate of the potential yield of poplar throughout the study area.
% Potential yield maps only take into consideration the climatic,
% environmental and phenological aspects.  They do not include the
% technical considerations of the technical feasibility of poplar as a
% crop.  However, because management regimes effect poplar growth rates
% over the period of the rotation different harvesting patterns are
% included in the potential growth models.  The two practices considered
% were 12 year cycled harvesting, and 4 four year coppicing cycles.

The potential growth model implemented the \ac{3pg} forest growth
model over the study area, by implementing the \ac{3pg} model within a
\ac{GIS}.  \ac{3pg} modeled the growth of the poplar over a 12 year
cycle, ether with or without the coppicing cycles.  

% Table~\ref{tab:3pg-grids} shows the required input grid parameters for
% the \ac{3pg} model.  In addtion, the model requires a number of other
% parameterizations, primarily for the modeling of the forest. Table

% \begin{table}[!ht]
% \caption{
% \textbf{Modeling Grids}}
% \begin{tabular}{|c|c|c|}
% \hline
% Parameter & Units & Source \\
% \hline
% elevation & \meter & \\
% \hline
% precipitation & \milli\meter & \\
% \hline
% temperature & \celsius & \cite{prism-temp} \\
% \hline
% humidity & \unit{\%}  & \\
% \hline
% radiation & \mega\joule\per\squaremetre\usk\dday &  \\
% \hline
% sub-freezing days & \dday  & \\
% \hline
% \end{tabular}
% \begin{flushleft}The table shows the data sources used for the
%   modeling of the potential Growth model for poplar.
% \end{flushleft}
% \label{tab:3pg-grids}
%  \end{table}

\begin{table}[!ht]
\caption{
\textbf{\ac{3pg} Model Constants}}
\begin{tabular}{|c|c|c|}
\hline
Parameter & Units & Source \\
\hline
Stand Age & \meter & PRISM~\cite{Amichev2010} \\
\hline
precipitation & \milli\meter & \cite{prism-precip} \\
\hline
\end{tabular}
\begin{flushleft}The table shows the data sources used for the
  modeling of the potential Growth model for poplar.
\end{flushleft}
\label{tab:3pg-grids}
 \end{table}

\subsection*{Poplar Hybrids}

\begin{table}[!ht]
\caption{
\textbf{Growth Model Source Data}}
\begin{tabular}{|c|c|c|}
table information
\end{tabular}
\begin{flushleft}Data sources used in the development of the model.
\end{flushleft}
\label{tab:data}
 \end{table}

\subsection*{Coppicing Model}
\label{sec:coppicing-model}
The original \ac{3pg} model allocates mass produced from transpiration
into the the creation of new roots, stems and foliage.  In the
standard \ac{3pg} model, the amount allocated to stems is basically a
constant allocation, with a modifier due to the stress the plantation
is under.  In a \ac{SWRC} however, coppicing the trees during
harvesting introduces a situation where the stem and foliage are
removed, but the root ball remains.  The \ac{3pg} model has no
mechanism to increase to start re-growth from this stituation, or
moderate the allocations based on the coppiced plant.

\begin{figure}[!ht]
\begin{subfigure}[b]{.1125\linewidth}
\centering
\includegraphics[width=1.0\linewidth]{img/tree_pics_1}
\caption{}  %\caption{planting}
\label{fig:grow_1}
\end{subfigure}
\begin{subfigure}[b]{.1125\linewidth}
\centering
\includegraphics[width=1.0\linewidth]{img/tree_pics_2}
\caption{}  %\caption{sprout}
\label{fig:grow_2}
\end{subfigure}
\begin{subfigure}[b]{.1125\linewidth}
\centering
\includegraphics[width=1.0\linewidth]{img/tree_pics_3}
\caption{}  %\caption{growth}
\label{fig:grow_3}
\end{subfigure}
\begin{subfigure}[b]{.1125\linewidth}
\centering
\includegraphics[width=1.0\linewidth]{img/tree_pics_4}
\caption{}  %\caption{maturity}
\label{fig:grow_4}
\end{subfigure}
\begin{subfigure}[b]{.1125\linewidth}
\centering
\includegraphics[width=1.0\linewidth]{img/tree_pics_5}
\caption{}  %\caption{coppiced}
\label{fig:grow_5}
\end{subfigure}
\begin{subfigure}[b]{.1125\linewidth}
\centering
\includegraphics[width=1.0\linewidth]{img/tree_pics_6}
\caption{}  %\caption{resprouting}
\label{fig:grow_6}
\end{subfigure}
\begin{subfigure}[b]{.1125\linewidth}
\centering
\includegraphics[width=1.0\linewidth]{img/tree_pics_7}
\caption{}  %\caption{regrowth}
\label{fig:grow_7}
\end{subfigure}
\begin{subfigure}[b]{.1125\linewidth}
\centering
\includegraphics[width=1.0\linewidth]{img/tree_pics_4}
\caption{}  %\caption{next harvest}
\label{fig:grow_8}
\end{subfigure}
\caption{ \textbf{Poplar \ac{SRWC} growth.} The growth stages for poplar
  grown as an \ac{SRWC} with one coppicing cycle shown. }
\label{fig:grow}
\end{figure}

Figure~\ref{fig:grow} shows a figure of the growth of a poplar planted
as a \ac{SRWC}.  Poplars are propagated via cuttings of bare poplar
stem, mostly under the soil,~(Figure~\ref{fig:grow_1}).  The cutting
provides energy for the establishment of the seedling via buds above
and below the surface~(Figure~\ref{fig:grow_2}). As the poplar
matures, all growth comes from the productivity provided by the
foliage.  This includes the establishment of the root
ball~(Figure~\ref{fig:grow_3}).By the time the poplar is ready for
coppicing~(Figure~\ref{fig:grow_4}), the plants are well established.

After coppicing, there is a surfeit of root mass, and no foilage to
provide photosynthetic based productivity to the
tree~(Figure~\ref{fig:grow_5}).  Later, resprouting begin with the
root ball providing initial energy for the restablishment of
seedlings~(Figure~\ref{fig:grow_6}) which then leads through the
maturation of the poplar, ready for the next
harvest~(Figures~\ref{fig:grow_7} and~\ref{fig:grow_8}).

We have developed a simple root interaction system to added to the
\ac{3pg} model to model the behavior both of the post-coppiced
situation as well as the initial planting of the cutting.  The model
basically allows for some additional production to come from the root
mass under certain conditions.  The seasonal timing of the
contribution is moderated by comparing the actual~\ac{NPP} to the
\ac{NPP} the root could produce if it had a more complete canopy.
This difference serves as a target for the root contribution.  This
method times regrowth with climatic conditions that are favorable for
growth.  The monthly root contribution affects the length of time for
the contribution to be made.  In poplar plantations, the initial
planting is modeled the same way, with a different set of parameters
to model the contribution from planted poplar stick.


\begin{figure}[!ht]
  \centering
  \includegraphics{img/tree_pics_10}
  \caption{\textbf{Coppice Model Overview}}
  \label{fig:coppice}
\end{figure}

\begin{align}
\acs{dW}=\acs{NPP}+\acs{RP} \\
\acs{RP} = \begin{cases} 0 & \acs{NPPres} <=0 \\
\acs{fR} ~ min (\acs{dRdef} ,\acs{NPPres}) & \acs{NPPres} > 0  
\end{cases} \\
\acs{NPPres} = \acs{NPPt}-\acs{NPP} \\
\acs{dRdef} = \acs{WR}(\acs{WR}/\acs{W} - \acs{pRx})\acs{Rdp}
\end{align}

\subsection*{Management regimes}
\label{sec:management-reg}
In wood energy plantations, the management regime impacts the time
between harvest, the fraction of the total biomass that enters the
supply chain, and growth rates. Coppice style management generally
suggests the use of a continuous harvest machine (forage harvester) at
3-5 year intervals over a 12-18 year rotation. Following coppice
harvest the stumps are left to re-sprout and will be harvested again
after 3-5 years.  Stem biomass that enters the feedstock supply chain
under coppice management is determined by the year of first entry
which determines stump volume.  Stump volume under coppice management
is not considered to enter the supply chain at any point in the
rotation. Round wood production implies the use of a piece-wise
(harvester) or semi piece-wise harvesting equipment
(feller-buncher). If stem diameter at harvest exceeds $\approx$ 8-10
cm, piece-wise harvesting is necessary. Following roundwood harvest,
the field will be cleared and preparedn for re-planting.

% Experimental work has suggested that there may be advantages to
% intercropping coppice and roundwood production. \emph{WHY?}. Under
% intercropping regime supply chain loss is determined by the combined
% loss from coppiced stumps and roundwood stumps over the rotation
% period.

\subsubsection*{Stump volume}
\label{sec:stump-volume}

The \ac{3pg} growth model estimates partitioning of biomass between stem,
leaf, and roots within a spatial domain defined by model input
parameters. Harvesting results in the collection of a fraction of the
total accumulated stem biomass into the feedstock supply chain for
energy production. The fraction of the biomass retained depends upon
the management regime. Stump, and saw kerf acount for the fraction of
stem biomass that is not captured in the supply chain.

Stump volume was determined using total tree volume predicted by \ac{3pg},
a stem taper function from for young poplar stands
\cite{Benbrahim2003} , and allometric relationships between total
biomass and diameter at breast height ($dbh$) \cite{Brahim2000}. The
taper equation was used to establish stem diameter at stump height,
and basal diameter, both of which which are necessary for determining
stump volume. Allometric biomass relationships in forestry are
generally described as in ~(\ref{form}).
\begin{equation}
  \label{eq:form}
  X=aY^b
\end{equation}

\input{stumpVol}

%\subsubsection{Growth impacts}
%\label{sec:growth-impacts}

\section*{Validation}

% Results and Discussion can be combined.
\section*{Results}

The \ac{3pg} model was tested with a number

\subsection*{Proe 2002}

\cite{proe02} described the results of a number of different \ac{SWRC}
field trials, including annual measurements of biomass, root:shoot
ratio, leaf/stem ratios, LAI, PAR and other parameters.  Light
interception was measured seasonally as well.  The studies were
developed over a 5 year period. Coppiced and single stem comparsions
were made, but only for the affects of early coppicing after one year.
Proe found increase in biomass production for more closely spaced
plantings still noticable up to the 5th year of the study.

Comparisons the the \ac{3pg} model where made by replicating the three
plantings under the conditions described.  Weather information for the
Scotland field plots were determined from ?.  Comparisons were made
between the \ac{3pg} model and the measured values, for spacings of 1m and
1.5m, and the single stem, and the coppicing methods.  Comparisons of
the total biomass, LAI, and light interception were made.

\begin{figure}
  \centering
  
  \caption{Woody and Stem biomass predictions vs. Measurements}
  \label{fig:proe-wood}
\end{figure}

\begin{figure}[!ht]
  \centering
  
  \caption{LAI and Light interception predictions vs. measurements}
  \label{fig:proe-light}
\end{figure}

\subsection{Pontailler 1999}
\label{sec:pont}

Pontailler, \cite{pontailler99biomass-yield} described the results of
biomass measurements over 5 two year coppicing events for a five
different poplar species.  Pontailler found consistant biomass returns
on the plots for the most part, and attributed differences, especially
the lower returns in 1991-1992, as due to drought conditions for the
region.  The \ac{3pg} model was assigned parametric variations due to
genotype.  These are shown in table~\ref{tab:pont-3pg}.

\begin{table}[!ht]
  \centering
%  \begin{tabular}{}    
%  \end{tabular}
  \caption{\ac{3pg} parameter variations of \ac{3pg} among genotypes}
  \label{tab:pont-3pg}
\end{table}

\begin{figure}[!ht]
  \centering

  \caption{Comparison of model to measurements for yearly growth over five
    coppicing events.}
\label{fig:pont-biomass}
\end{figure}


\subsection{Afas 2008}
\label{Afas2008}

Starting in 1996, Afas,~\cite{Afas2008a} studied 17 types of poplar
over the course of 11 years for a field study in Belguim.  The study
included measurements of three seperate coppicing events.  The study
found relatively high mortality rates for some of the genotypes.
Comparisons between the \ac{3pg} and measured values were made for the
genotypes with a survival rate of over 85\%.  No measurements of below
ground biomass were included.  Afas did propose an alimetric
realtionship for non-destructive biomass estimations based on diameter
measurements of the stems from the stool.

Variations among the genotypes were estimated with changes to the \ac{3pg}
input parameters of : ?

\begin{table}[!ht]
  \centering
%  \begin{tabular}{}
    
%  \end{tabular}
  \caption{\ac{3pg} parameter variations of \ac{3pg} among genotypes}
  \label{tab:afas-3pg}
\end{table}

\begin{figure}[!ht]
  \centering

  \caption{Comparison of model to measurements for yearly growth over three
    coppicing events.}
\label{fig:afas-biomass}
\end{figure}

\subsection*{Stocking Density}
\label{sec:stocking-density}

Initial stocking density is an important consideration, as planting
density is a primary driver in initialization costs, and need to be
recoupped with gains in biomass production.  \ac{SRWC} differ from
more traditional plantations in the the total biomass is important
rather than the biomass per stem.  Actually, the fact that coppicing
results in more stems per stump can lead to efficiencies in the
harvesting 

\cite{proe02} compare the results of two 

To compare comparison with the
model, it's 

There have been a few studies


%\section*{Discussion}

% Do NOT remove this, even if you are not including acknowledgments
\section*{Acknowledgments}
This project was funded by the USDA's Advanced Hardwood Biofuels for
the Pacific Northwest project, Number \#.

%\section*{References}
% The bibtex filename
\bibliography{ahb-pnw}

\section*{Figure Legends}

Move Figures here at the end.

\section*{Tables}

Move Table here at the end.

%% \begin{table}[!ht]
%% \caption{
%% \textbf{Title}}
%% \begin{tabular}{|c|c|c|}
%% table information
%% \end{tabular}
%% \begin{flushleft}Caption
%% \end{flushleft}
%% \label{tab:}
%%  \end{table}

\end{document}

