% Template for PLoS
% Version 1.0 January 2009
%
% To compile to pdf, run:
% latex plos.template
% bibtex plos.template
% latex plos.template
% latex plos.template
% dvipdf plos.template

\documentclass[10pt]{article}

% amsmath package, useful for mathematical formulas
\usepackage{amsmath}
% amssymb package, useful for mathematical symbols
\usepackage{amssymb}

% graphicx package, useful for including eps and pdf graphics
% include graphics with the command \includegraphics
\usepackage{graphicx}

% cite package, to clean up citations in the main text. Do not remove.
\usepackage{cite}

\usepackage{color} 
\usepackage{units}

% Use doublespacing - comment out for single spacing
%\usepackage{setspace} 
%\doublespacing


% Text layout
\topmargin 0.0cm
\oddsidemargin 0.5cm
\evensidemargin 0.5cm
\textwidth 16cm 
\textheight 21cm

% Bold the 'Figure #' in the caption and separate it with a period
% Captions will be left justified
\usepackage[labelfont=bf,labelsep=period,justification=raggedright]{caption}

% Use the PLoS provided bibtex style
\bibliographystyle{plos2009}

% Remove brackets from numbering in List of References
\makeatletter
\renewcommand{\@biblabel}[1]{\quad#1.}
\makeatother


% Leave date blank
\date{}

\pagestyle{myheadings}
%% ** EDIT HERE **


%% ** EDIT HERE **
%% PLEASE INCLUDE ALL MACROS BELOW

%% END MACROS SECTION

\begin{document}

% Title must be 150 characters or less
\begin{flushleft}
{\Large
\textbf{Modeling Poplar Growth for the Pacific Northwest}
}
% Insert Author names, affiliations and corresponding author email.
\\
Quinn Hart$^{1,\ast}$, 
Peter Tittmann$^{2}$, 
Bryan Jenkins$^{2}$
\\
\bf{1} Department of Land, Air, and Water, University of Califonia, Davis, USA
\\
\bf{2} Energy Institute, University of Califonia, Davis, USA
%\\
%\\bf{3} Author3 Dept/Program/Center, Institution Name, City, State, Country
\\
$\ast$ E-mail: qjhart@ucdavis.edu
\end{flushleft}

% Please keep the abstract between 250 and 300 words
\section*{Abstract}

% Please keep the Author Summary between 150 and 200 words
% Use first person. PLoS ONE authors please skip this step. 
% Author Summary not valid for PLoS ONE submissions.   
\section*{Author Summary}

\section*{Introduction}

% Results and Discussion can be combined.
\section*{Results}
\subsection*{Potential Growth}
\subsection*{Technical Availibility}


%\section*{Discussion}

% You may title this section "Methods" or "Models". 
% "Models" is not a valid title for PLoS ONE authors. However, PLoS ONE
% authors may use "Analysis" 
\section*{Models}

Figure~\ref{fig:study-area} shows the area of study.

Tabl

\subsection*{Potential Growth Model}

Figure~\ref{fig:growth-model} shows the modeling path used in the
development of the poplar growth model. The model developed shows the
total potential of the grieed

\subsection*{Poplar hybrids}

Table 

\subsection*{Technical Availibility Model}


% Do NOT remove this, even if you are not including acknowledgments
\section*{Acknowledgments}
This project was funded by the USDA's Advanced Hardwood Biofuels for the Pacific Northwest project, #.

%\section*{References}
% The bibtex filename
\bibliography{template}

\section*{Figure Legends}
\begin{figure}[!ht]
\begin{center}
\vspace*{4cm}
%\includegraphics[width=4in]{figure_name.2.eps}
\end{center}
\caption{ {\bf Study Area.}  Poplar Growth modeling was developed for
  a larger region of the Pacific Northwest.  The model was run on a
  standard grid, shown by the rectangle. Grid boundaries where run at
  \unit[2]{$km^2$} and \unit[8]{$km^2$}.  }
\label{fig:study-area}
\end{figure}

\begin{figure}[!ht]
\begin{center}
\vspace*{4cm}
\end{center}
\caption{ {\bf Potential Growth Modeling Workflow.} This graph shows the processing steps used to develop the potential growth model }
\label{fig:growth-model}
\end{figure}

\begin{figure}[!ht]
\begin{center}
\vspace*{4cm}
\end{center}
\caption{ {\bf Potential Poplar Growth.} This shows the potential popular growth for a number of hybrids.  }
\label{fig:growth-map}
\end{figure}


\begin{figure}[!ht]
\begin{center}
\vspace*{4cm}
\end{center}
\caption{ {\bf Technical Modeling Workflow.} This graph shows the determination of the availability of lands for poplar plantations.  }
\label{fig:tech-model}
\end{figure}

\begin{figure}[!ht]
\begin{center}
\vspace*{4cm}

\end{center}
\caption{ {\bf Technical Hardwood Availability}  }
\label{fig:tech-map}
\end{figure}


\section*{Tables}
\begin{table}[!ht]
\caption{
\bf{Modeling Grids}}
\begin{tabular}{|c|c|c|}
table information
\end{tabular}
\begin{flushleft}Caption
\end{flushleft}
\label{tab:}
 \end{table}

\begin{table}[!ht]
\caption{
\bf{Growth Model Source Data}}
\begin{tabular}{|c|c|c|}
table information
\end{tabular}
\begin{flushleft}Data sources used in the development of the model.
\end{flushleft}
\label{tab:data}
 \end{table}

\begin{table}[!ht]
\caption{
\bf{Technical Availability Source Data}}
\begin{tabular}{|c|c|c|}
table information
\end{tabular}
\begin{flushleft}Caption
\end{flushleft}
\label{tab:}
 \end{table}

%% \begin{table}[!ht]
%% \caption{
%% \bf{Title}}
%% \begin{tabular}{|c|c|c|}
%% table information
%% \end{tabular}
%% \begin{flushleft}Caption
%% \end{flushleft}
%% \label{tab:}
%%  \end{table}

\end{document}

